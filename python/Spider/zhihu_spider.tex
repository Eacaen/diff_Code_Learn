%# -*- coding: utf-8 -*-
%!TEX encoding = UTF-8 Unicode
\documentclass{article}
\usepackage{titlesec}
\titleformat*{\section}{\centering}
\usepackage[UTF8]{ctex}
    \title{爬虫}
\begin{document}
   \maketitle
   \section*{\Large 知乎---爬虫}
   \subsection{模拟登录} 
  \begin{enumerate}
    \item 知乎爬取时必须带上header。
    \item \_xref 可以从登录网页的源码中提取,但是提取和登陆的网页必须是同一个。urllib2.urlopen 是打开可能不同的,可以使用requests库建立session。
  \end{enumerate}

    \subsection{爬取 follow , 问题等内容}
    【部分知乎问题已有答案了】

    知乎网页是动态加载问题,怎么解决?
    以爬取问题或关注人为例,通过开发者工具中的Network -- XHR / Js 选项,向下拉取页面,可以看出网页在加载时是通过  offset 和  start\_offset 来控制起始的爬取网页,
     来获取新的内容(在XHR中网页不是在类似batch的请求中,\textbf{注意:} 如果要爬取全部内容  start\_offset = 0 而不是  3,offset  与显示的不同),然后复制出URL备用。
\\

    编写header, 除了基本header外,必须还要有 authorization ,authorization貌似在不同的爬取需求中都是相同的,但是header必须包含它。打开网页后的内容好像是 json 格式, 使用 json.loads(url.content) 将内容转化为字典形式。
\\

    通过正则表达式 或是 BeautifulSoup 可以提取需要的内容,在提取速度和存储方式上,文件操作等方面有待加强。
      
    
\end{document} 