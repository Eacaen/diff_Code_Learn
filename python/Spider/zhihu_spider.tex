%# -*- coding: utf-8 -*-
%!TEX encoding = UTF-8 Unicode
\documentclass{article}
\usepackage{titlesec}
\usepackage{graphicx}
\usepackage{pythonhighlight}


\titleformat*{\section}{\centering}
\usepackage[UTF8]{ctex}
    \title{爬虫}
\begin{document}
   \maketitle
   \section*{\Large 知乎---爬虫}
   \subsection{模拟登录} 
  \begin{enumerate}
    \item 知乎爬取时必须带上header。
    \item \_xref 可以从登录网页的源码中提取,但是提取和登陆的网页必须是同一个。urllib2.urlopen 是打开可能不同的,可以使用requests库建立session。
  \end{enumerate}

    \subsection{爬取 follow , 问题等内容}
    【部分知乎问题已有答案了】

    知乎网页是动态加载问题,怎么解决?
    以爬取问题或关注人为例,通过开发者工具中的Network -- XHR / Js 选项,向下拉取页面,可以看出网页在加载时是通过  offset 和  start\_offset 来控制起始的爬取网页,
     来获取新的内容(在XHR中网页不是在类似batch的请求中,\textbf{注意:} 如果要爬取全部内容  start\_offset = 0 而不是  3,offset  与显示的不同),然后复制出URL备用。
\\

    编写header, 除了基本header外,必须还要有 authorization ,authorization貌似在不同的爬取需求中都是相同的,但是header必须包含它。打开网页后的内容好像是 json 格式, 使用 json.loads(url.content) 将内容转化为字典形式。
\\

    通过正则表达式 或是 BeautifulSoup 可以提取需要的内容,在提取速度和存储方式上,文件操作等方面有待加强。
      
      \subsection{多线程 }
      \subsubsection{反爬措施防止被屏蔽}
      使用多线程多次爬取后知乎服务器貌似会屏蔽访问,不知道是  IP 的问题还是什么,如果还能在浏览器正常打开知乎的话可以采用更换 headers 中 authorization 的方法进行访问。

      或是使用IP代理?

       \subsubsection{多线程的顺序问题}
        多线程或多进程由于是并发,在访问网页时不能按 Page 顺序访问,在下载排序命名时也会顺序错乱。 直接采用 \textbf{全局变量}并加\textbf{多线程,多线进锁},这样虽然会影响 Class 的 独立性 但是可以使得顺序不错乱且按照顺序保存。

       【 标号顺序不乱但是内容顺序仍有可能混乱,不一定的按照网页内容了。】

     \subsection{多进程}
          \subsubsection{Windows 下多进程无法运行}
              \textbf{SAD $-\_-$}

              多进程程序在Windows下仍无法运行,在Linux下运行正常。
          \subsubsection{多进程全局变量共享}
              global 变量和Queue模块中的Queue() 在全局变量中貌似不起作用,使用 multiprocessing 中的 各个模块来在进程中同步信号。
                \begin{python}
from multiprocessing import Process, JoinableQueue,Lock ,Queue , Pool , Value
counter = Value('i', 0)
counter.value += 1  
                \end{python}

            \subsubsection{使用进程池可以退出}
                当爬取网页太多时,进程会也不停止也不报错。 尝试使用进程池可以正常退出。
                \begin{python}
pool  = Pool( multiprocessing.cpu_count() )
for i in xrange( 2 * multiprocessing.cpu_count() ) :
        pool.apply_async( func , args = ( ) )
pool.close()  
pool.join() 
                \end{python}


    \section*{\Large Scrapy---爬虫}
    \subsection{Windows 下安装问题}
    windows 下 anaconda 安装会出现

    \begin{python}
from cryptography.hazmat.bindings._openssl import ffi, lib
ImportError: DLL load failed
    \end{python}: 操作系统无法运行 \%1。

     删除了window/system32/ 的 libeay32.dll和ssleay32.dll,不过我的建议是重命名,直接后面加 {\_bak} 就好,免得出什么问题。

    \subsection{Scrapy}
      \begin{enumerate}
            \item Scrapy 是一种爬虫框架 / 爬虫引擎, 你需要在里面填上爬虫函数,并对数据下载,提取,处理, 存储。
            \item Scrapy 中所以爬虫都继承自  scrapy.spiders.Spider。另外有别的作用更加细分的爬虫如 Crawlspider 通过设置自身的 rules 从爬去的网页中提取URL 继续爬取。
            \item Scrapy 调试在 Shell 中, 开发的话应该掌握。Xpath() 语法用于对数据进行提取,类似正则。
            \item items.py 中定义各个数据的存储容器。pipeline.py 进行对item 数据的查重,丢弃,验证,保存工作。
      \end{enumerate}

    \subsection{Tips of Scrapy}
      \subsubsection{不使用命令行启动}
          在根目录下创建 main.py 
          \begin{python}
from scrapy import cmdline
cmdline.execute('scrapy crawl Spider_name'.split())
          \end{python}

     \subsubsection{ 爬虫开始,结束时传递参数 }
          Scrapy 默认在开始结束时运行:
          \begin{python}
def open_spider(self, spider):
def close_spider(self, spider):
          \end{python}

          与Spider.py 中的 Super 结合使用
          \begin{python}
super(Spider_name,  self).__init__()
          \end{python}


\end{document} 